\documentclass[11pt,a4paper,notitlepage]{report}
\usepackage[latin1]{inputenc}
\usepackage{amsmath}
\usepackage{amsfonts}
\usepackage{amssymb}
\usepackage{graphicx}
\author{160007345}
\title{Final Report}
\begin{document}
	\maketitle
	\section*{Rough notes}
	Title page
	Containing the title of the project, the names of the
	student(s), "University of St Andrews" and the date of
	submission. You may add the name of your supervisor
	if you wish.
	
	\section*{Abstract}
	Abstract Outline of the project using at most 250 words.
	
	\section*{Declaration}
	Declaration
	"I/we* declare that the material submitted for
	assessment is my/our* own work except where credit is
	explicitly given to others by citation or
	acknowledgement. This work was performed during
	the current academic year except where otherwise
	stated.
	"The main text of this project report is NN,NNN*
	words long, including project specification and plan.
	"In submitting this project report to the University of
	St Andrews, I/we* give permission for it to be made
	available for use in accordance with the regulations of
	the University Library. I/we* also give permission for
	the title and abstract to be published and for copies of
	the report to be made and supplied at cost to any bona
	fide library or research worker, and to be made
	available on the World Wide Web. I/we* retain the
	copyright in this work."
	(* to be filled in as appropriate)
	If there is a strong case for the protection of
	confidential data, the parts of the declaration giving
	permission for its use and publication may be omitted
	by prior permission of the Honours Coordinator.
	1
	2017/18
	
	\section*{Contents Page}
	Contents page Table of contents
	
	\section*{Introduction}
		Introduction
	Describe the problem you set out to solve and the
	extent of your success in solving it. You should include
	the aims and objectives of the project in order of
	importance and try to outline key aspects of your
	project for the reader to look for in the rest of your
	report.
	
	\section*{Context Survey}
	Context survey
	Surveying the context, the background literature and
	any recent work with similar aims. The context survey
	describes the work already done in this area, either as
	described in textbooks, research papers, or in publicly
	available software. You may also describe potentially
	useful tools and technologies here but do not go into
	project-specific decisions.
	\subsection*{DMX, SACN and ACN}
	\subsection*{Other protocols}
	
	\section*{Requirement specification}
	Requirements
	specification
	Capturing the properties the software solution must
	have in the form of requirements specification. You
	may wish to specify different types of requirements and
	given them priorities if applicable.
	
	\section*{Software Engineering Process}
		Software
	engineering
	process
	The development approach taken and justification for
	its adoption.
	
	\section*{Ethics}
	Ethics
	Any ethical considerations for the project. You should
	scan the signed ethical approval document, and include
	it as an appendix.
	
	\section*{Design}
	Design
	Indicating the structure of the system, with particular
	focus on main ideas of the design, unusual design
	features, etc.
	\subsection*{ANSI E1.31-2018}
	\subsection*{Critique of the protocol}
	\section*{Implementation}
	Implementation
	How the implementation was done and tested, with
	particular focus on important / novel algorithms
	and/or data structures, unusual implementation
	decisions, novel user interface features, etc.
	
	\subsection*{Implementation dependent specifics}
	
	\section*{Evaluation and Critical Appraisal}
	Evaluation and
	critical
	appraisal
	You should evaluate your own work with respect to
	your original objectives. You should also critically
	evaluate your work with respect to related work done
	by others. You should compare and contrast the project
	to similar work in the public domain, for example as
	written about in published papers, or as distributed in
	software available to you.
	
	\section*{Conclusions}
	Conclusions
	You should summarise your project, emphasising your
	key achievements and significant drawbacks to your
	work, and discuss future directions your work could be
	taken in.
	
	\section*{Appendices}
	The appendices to your report will normally be as follows.
	Testing
	summary
	This should describe the steps taken to debug, test,
	verify or otherwise confirm the correctness of the
	various modules and their combination.
	
	\section*{Testing}
	\subsection*{Automated Testing}
	\subsection*{Real-world Testing}
	
	\section*{User Manual}
	User manual Instructions on installing, executing and using the
	system where appropriate.
	
	\section*{Other Appendices}
	Other
	appendices
	If appropriate, you may include other material in
	appendices which are not suitable for inclusion in the
	main body of your report, such as the ethical approval
	document.
	You should not include software listings in your project report, unless it is
	appropriate to discuss small sections in the main body of your report. Instead,
	you will submit via MMS your code and associated material such as JavaDoc
	documentation and detailed UML diagrams
	
\end{document}