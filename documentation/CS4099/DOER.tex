\documentclass[12pt,a4paper,notitlepage]{report}
\usepackage[utf8]{inputenc}
\usepackage{amsmath}
\usepackage{amsfonts}
\usepackage{amssymb}
\usepackage{graphicx}
\usepackage{fourier}
\usepackage[left=2cm,right=2cm,top=2cm,bottom=2cm]{geometry}
\author{Paul Lancaster}
\title{CS4099 DOER}
\begin{document}
	
	
 It
generally consists of about two pages of text and must include these four sections:
Description
The title and a short description of the project aims,
context and background. It should explain the big
picture of what you would like to achieve, why it is
important, and how you intend to go about doing it (e.g.
by using some kind of technology or developing a new
algorithm, or following a particular methodology, etc.)
Objectives
This is a list of clearly defined, measurable goals you
intend to achieve by the end of your project. This could
include any software artefacts you intend to submit in
the end, results of an evaluation (for surveys or research
algorithms), etc. Your performance will be measured
against these objectives.
Typically, you will list about 3-5 primary objectives
which are necessary for a project to be deemed
successful, and further 3 or so secondary objectives
which allow a successful project to be extended in an
interesting direction. Occasionally, tertiary objectives
may also be listed, but these are comparatively rare.
Ethics
Here you should discuss any ethical considerations
pertaining to your project. Start with the selfassessment form from the Student Handbook (Ethics
section). If you can answer “No” to all questions on the
self-assessment form, this section of the DOER
document will be brief and state that there are no ethical
considerations.
If you are planning to work with people (especially
children), animals, sensitive private data, or if there are
other considerations, you should discuss them here, and
explain how you went about obtaining necessary
approval (any Ethics applications).
The self-assessment form and any other relevant
documents (if applicable) should be scanned and
uploaded to the “Ethics” slot on MMS.
1
2017/18
Resources
This is a list of any special resources your project will
need: hardware, software, licenses, access to
infrastructure (e.g. compute servers), drones, etc. Think
ahead, but be realistic -- the School will not be able to
fulfill all requests.
Most projects can be completed using standard school
equipment, in which case this section will contain only a
short statement confirming this.
You and your supervisor will have to agree on everything in the DOER document.
Typically, the process looks like this:
1. Schedule a meeting with the supervisor to flesh out the description,
objectives, and any needed resources,
2. Write this all up in a word processor, following the structure presented
above,
3. Make sure both you and the supervisor agree about the contents (via email
or in person),
4. Submit the DOER to MMS.
\end{document}