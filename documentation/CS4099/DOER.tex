\documentclass[12pt,a4paper,notitlepage]{report}
\usepackage[utf8]{inputenc}
\usepackage{amsmath}
\usepackage{amsfonts}
\usepackage{amssymb}
\usepackage{graphicx}
\usepackage{fourier}
\usepackage[left=2cm,right=2cm,top=2cm,bottom=2cm]{geometry}
\author{Paul Lancaster}
\title{CS4099 DOER: Rust implementation of the sACN protocol}
\begin{document}
	\maketitle
\section*{Description}
\subsection*{Aim}
The project aims to expand an existing implementation \cite{ORIGNIAL_IMPL} of the streaming architecture for control networks (sACN) protocol \cite{ANSI_E1.17} in rust so that it fulfils the scope of the protocol as defined in \cite{ANSI_E1.31}. While open source implementations of the sACN protocl exist in other languages \cite{C_IMPL} there are no fully implemented versions in rust (as of starting this project). 
\subsection*{Background}
\subsubsection*{DMX512} DMX512 is an protocol used in the entertainment industry for the control of lighting, effects and other devices. It works by daisy chaining devices together into distinct physical chains (called universes) and is a one way protocol. This means that the devices in the line cannot communicate their presence back to the controller so the controller must know about the devices ahead of time and their addresses so it can broadcast packets down the line which the devices then receive and use. The DMX packets are a fixed size and contain five hundred and twelve 8-byte channel (+ a start code) which allows them to control up to 512 different devices on a singular line. A device may support the use of multiple channels to control different functionalities so for example a light with RGB colour mixing may use 3 channels to allow control of the Red, Green and Blue individually. Since there are only 512 channels available on a single universe this quickly imposes a limitation to the number of devices that can be connected together, especially as modern lighting fixtures commonly use upwards of 30 channels each for a moving light with usage of many more not uncommon. The solution to this was previously to simply have more physical lines (universes) and in this way allow more devices to be controlled simultaneously. This comes with a number of problems however as each new physical line means a new cable coming directly from the control desk.

\paragraph*{DMX512 Problems}
\begin{list}{}{}
	\item As the control desk is often far from the devices themselves (at the back of the venue whereas the lights/devices are above the stage) it means that many cables need to be run which can be expensive and time consuming.
	\item The length of the cable runs can cause signal interference / degradation and DMX as a 1 way protocol does not have any error correction (bad frames if detected are thrown out).
	\item The protocol only allowing 512 channels per physical line means that a device cannot have more channels than this. This is particularly a problem recently with the advent of complex fixtures which may have many LED's with individual colour control.
\end{list}

\subsubsection*{sACN}
One solution to solve some of the problems with DMX is to send it using UDP over a standard IP based network and one of the protocols created to do this is sACN. This allows many DMX packets (and so many universes) to be simultaneously sent using a single network cable from the console and then to be received by the devices. Often for backwards compatibility reasons the sACN is converted back into DMX packets before being sent to the device as most devices older than a few years do not support direct sACN communication but this is rapidly increasing - particularly with higher end professional fixtures. 

\subsubsection*{Rust}
Rust \cite{RUST_LANG} is a language designed to be used for writing fast, memory efficient systems code. It has relatively low overhead and because it doesn't require the additional overhead of a garbage collector it is perfect for usage in embedded or high performance systems. This makes it an ideal language for usage with sACN as responding quickly to control events is vital to a lighting system (imagine a sound effect goes off but the lighting doesn't until a second later). 

\section*{Objectives}
Objectives
This is a list of clearly defined, measurable goals you
intend to achieve by the end of your project. This could
include any software artefacts you intend to submit in
the end, results of an evaluation (for surveys or research
algorithms), etc. Your performance will be measured
against these objectives.
Typically, you will list about 3-5 primary objectives
which are necessary for a project to be deemed
successful, and further 3 or so secondary objectives
which allow a successful project to be extended in an
interesting direction. Occasionally, tertiary objectives
may also be listed, but these are comparatively rare.

\section*{Ethics}
This project has no ethical considerations that require notification in this section.

\section*{Resources}
For testing the protocol and verifying it works with real world devices this project will require access to a control device capable of generating sACN packets and a device able to receive sACN packets and use them. I have access to these devices already as well as all equipment needed to connect them and so I do not require any equipment from the CS department. 
	
\begin{thebibliography}{9}
	\bibitem{ANSI_E1.17}
	ANSI E1.17 - 2015 Entertainment Technology—Architecture for Control Networks
	\bibitem{ORIGNIAL_IMPL}
	https://github.com/lschmierer/sacn
	\bibitem{ANSI_E1.31}
	ANSI E1.31 — 2018 Entertainment Technology Lightweight streaming protocol for transport of DMX512 using ACN
	\bibitem{DMX_INFO}
	https://www.element14.com/community/groups/open-source-hardware/blog/2017/08/24/dmx-explained-dmx512-and-rs-485-protocol-detail-for-lighting-applications (17/09/2019)
	\bibitem{C_IMPL}
	https://github.com/hhromic/libe131 (17/09/2019)
	\bibitem{RUST_LANG}
	https://www.rust-lang.org/ (17/09/2019)
	
\end{thebibliography}
	
	

Ethics
Here you should discuss any ethical considerations
pertaining to your project. Start with the selfassessment form from the Student Handbook (Ethics
section). If you can answer “No” to all questions on the
self-assessment form, this section of the DOER
document will be brief and state that there are no ethical
considerations.
If you are planning to work with people (especially
children), animals, sensitive private data, or if there are
other considerations, you should discuss them here, and
explain how you went about obtaining necessary
approval (any Ethics applications).
The self-assessment form and any other relevant
documents (if applicable) should be scanned and
uploaded to the “Ethics” slot on MMS.
1
2017/18
Resources
This is a list of any special resources your project will
need: hardware, software, licenses, access to
infrastructure (e.g. compute servers), drones, etc. Think
ahead, but be realistic -- the School will not be able to
fulfill all requests.
Most projects can be completed using standard school
equipment, in which case this section will contain only a
short statement confirming this.
You and your supervisor will have to agree on everything in the DOER document.
Typically, the process looks like this:
1. Schedule a meeting with the supervisor to flesh out the description,
objectives, and any needed resources,
2. Write this all up in a word processor, following the structure presented
above,
3. Make sure both you and the supervisor agree about the contents (via email
or in person),
4. Submit the DOER to MMS.
\end{document}